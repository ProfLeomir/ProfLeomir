% Modelo de Aula em LaTeX
% IFRS - Ibirubá
% Se utilizar o Overleaf, clique em RICH TEXT acima para uma visualização mais suave do código!

\documentclass[a4paper,11pt]{article}

%-------------------------------------------------------------------------
% PACOTES (não é necessário alterar nada) 
%-------------------------------------------------------------------------

\usepackage[utf8]{inputenc}
\usepackage[brazil]{babel} 
\usepackage{graphicx}
\usepackage{amsfonts}
\usepackage{color}
\usepackage{mdframed}
\usepackage{hyperref} 
\usepackage{qrcode}
\usepackage{geometry} 
\usepackage{pgfplots}
\usepackage{amsmath,amsthm,amsfonts,amssymb,amscd,amsxtra} 
\usepackage{multicol}
\usepackage{multirow}
\usepackage{enumerate}
\usepackage{tikz}
\usetikzlibrary{positioning,shapes,fit,arrows,through,calc,shapes.geometric,patterns,decorations.markings}
\usepackage{pgf,pgfplots}
\usepgfplotslibrary{fillbetween}
\pgfplotsset{compat=1.15}
\usepackage{mathrsfs}
\definecolor{myblue}{RGB}{56,94,141}
\definecolor{mygreen}{RGB}{51,153,0}
\definecolor{myred}{RGB}{204,0,0}
\usepackage{setspace}
\geometry{a4paper,left=1.5cm,right=1.5cm,top=1.5cm,bottom=1.5cm}

%-------------------------------------------------------------------------
% CABEÇALHO (não é necessário alterar nada) 
%-------------------------------------------------------------------------

\begin{document}

\pagestyle{empty}

\begin{minipage}{6cm} %Logo do IFPR
\resizebox{6cm}{1.6cm}{
\begin{tikzpicture}[square/.style={regular polygon,regular polygon sides=4}]
\tikzset{mynode/.style={square,rounded corners,draw=mygreen, top color=mygreen, bottom color=mygreen,very thick, text centered},}  
\node[circle, draw=myred , top color=myred, bottom color=myred] (q1) {$\quad$};
\node[mynode, below=0.08cm of q1] (q2) {$\;\;$};
\node[mynode, below=0.08cm of q2] (q3) {$\;\;$};
\node[mynode, below=0.08cm of q3] (q4) {$\;\;$};
\node[mynode, right=0.08cm of q1] (q5) {$\;\;$};
\node[mynode, below=0.08cm of q5] (q6) {$\;\;$};
\node[mynode, below=0.08cm of q6] (q7) {$\;\;$};
\node[mynode, below=0.08cm of q7] (q8) {$\;\;$};
\node[mynode, right=0.08cm of q5] (q9) {$\;\;$};
\node[mynode, right=0.08cm of q7] (q10) {$\;\;$};
\node[right=0.08cm of q8] (aux) {};
\node[text centered, right=0.06cm of q10] (q11) {\huge \sf \textbf{INSTITUTO FEDERAL}};
\node[mygreen, text centered, right=0.5cm of aux] (q12) {\huge\textsf{Rio Grande do Sul}};
\end{tikzpicture}}
\end{minipage}
\begin{minipage}{7cm}
\begin{center}
\begin{footnotesize}
%%%%%%%%%%%%%%%%%%%%%%%%%%%%%%%%%%%%%%%%%%%%%%%%%%%
%%%%%%%%%%%%%%%%%%%%%%%%%%%%%%%%%%%%%%%%%%%%%%%%%%%
%%%%%%%%%%%%%%%%%%%%%%%%%%%%%%%%%%%%%%%%%%%%%%%%%%%
{\Large \texttt{Exame}\\
%%%%%%%%%%%%%%%%%%%%%%%%%%%%%%%%%%%%%%%%%%%%%%%%%%%
%%%%%%%%%%%%%%%%%%%%%%%%%%%%%%%%%%%%%%%%%%%%%%%%%%%
%%%%%%%%%%%%%%%%%%%%%%%%%%%%%%%%%%%%%%%%%%%%%%%%%%%
\end{footnotesize}
\end{center}
\end{minipage}
\begin{minipage}{4cm}
\centering
\includegraphics[scale=0.12]{meBrasil.png}\\
\begin{scriptsize}
\textsf{Ministério da Educação}
\end{scriptsize}
\end{minipage}
\doublespacing
\begin{center}

\textbf{Engenharia Mecânica}  \ $-$ \ \textbf{Valor: 10,0}  \\

\textbf{Álgebra Linear e Geometria Analítica}  \ $-$ \ \textbf{Professor Me. Leomir A. S. Grave}   \\

\end{center}

\textbf{Nome:} \rule{12cm}{0.2mm} Data: 19/07/2023

%%%%%%%%%%%%%%%%%%%%%%%%%%%%%%%%%%%%%%%%%%%%%%%%%%%%
%%%%%%%%%%%%%%%%%% Escreva a Aula a partir daqui %%%
%%%%%%%%%%%%%%%%%%%%%%%%%%%%%%%%%%%%%%%%%%%%%%%%%%%%

\begin{mdframed}[backgroundcolor=gray!20]
\textbf{Questão 1} -- \textit{Peso: 1,0}
\end{mdframed}

Resolva, em $\mathbb{R}$, a equação:


\[
\begin{array}{|ccc|}
4x & 5 & - 3  \\
0 &  1 & - 1  \\
3x &  1 &  0 
\end{array}=1
\]


\begin{mdframed}[backgroundcolor=gray!20]
\textbf{Questão 2} -- \textit{Peso: 1,0}
\end{mdframed}


Seja a matriz $A=\begin{pmatrix}
2 & 5 & - 3  \\
0 &  1 & - 1  \\
2 &  1 &  0 
\end{pmatrix}$ 
e 
$B=\begin{pmatrix}
0 & 2 & - 1  \\
2 &  1 & - 2  \\
0 &  1 &  -5 
\end{pmatrix}
$\\
Determine\\
$$2\cdot A-3\cdot B$$

\begin{mdframed}[backgroundcolor=gray!20]
\textbf{Questão 3} -- \textit{Peso: 1,0}
\end{mdframed}

Resolva e classifique o sistema linear abaixo:

\begin{equation*}
\left\{
\begin{matrix}
2x +  y  -  2z  =  10\\
x  +  y   =  3\\
5x  +  4y  =  13
\end{matrix}
\right.
\end{equation*} 



\begin{mdframed}[backgroundcolor=gray!20]
\textbf{Questão 4} -- \textit{Peso: 1,0}
\end{mdframed}

Sejam dados os vetores $\vec u=(1,3,-2)$ e $\vec v=(2,-1,4)$, determine
$$2\cdot \vec u + (\vec v)^2$$

\begin{mdframed}[backgroundcolor=gray!20]
\textbf{Questão 5} -- \textit{Peso: 1,0}
\end{mdframed}

Calcule a soma de dois vetores, sendo $|\vec{F_1}|=8$, $|\vec{F_2}|=4$ e o ângulo entre eles $\alpha=60^{\circ}$.
\\
\\
\\

\begin{mdframed}[backgroundcolor=gray!20]
\textbf{Questão 6} -- \textit{Peso: 1,0}
\end{mdframed}

Dados $\vec{s}=(0, 2, -2)$ e $\vec{t}=(3, 0, 3)$, determine o ângulo entre esses dois vetores. 
\\
\\
\\

\begin{mdframed}[backgroundcolor=gray!20]
\textbf{Questão 7} -- \textit{Peso: 1,0}
\end{mdframed}

Verifique se o conjunto $W$ é um subespaço vetorial de $\mathbb{R}^2$, sendo $W=\{(x,y):x=y+5\};$ 
\\
\\
\\

\begin{mdframed}[backgroundcolor=gray!20]
\textbf{Questão 8} -- \textit{Peso: 1,0}
\end{mdframed}

Os vetores $u_1 = (2, 3)$, $u_2 = (3, \dfrac{9}{2})$ são linearmente independentes (LI) ou linearmente dependentes (LD)?
\\
\\
\\

\begin{mdframed}[backgroundcolor=gray!20]
\textbf{Questão 9} -- \textit{Peso: 1,0}
\end{mdframed}

Verifique se $f: \mathbb{R}^2 \longrightarrow \ \mathbb{R}^2, \ f(x,y)=(x-7,2y+3)$ é uma transformação linear.
\\
\\
\\

\begin{mdframed}[backgroundcolor=gray!20]
\textbf{Questão 10} -- \textit{Peso: 1,0}
\end{mdframed}

Verifique se o vetor $(5,7)$ pertence ao conjunto $Im(f)$, sendo $f: \mathbb{R}^2 \longrightarrow \ \mathbb{R}^2, \ f(x,y)=(2x-3y, x+5y)$.\\


%%%%%%%%%%%%%%%%%
%%%% QR Code %%%%
%%%%%%%%%%%%%%%%%
\begin{minipage}[t]{0.6\textwidth}
\end{minipage}
\hfill
\begin{minipage}[t]{0.3\textwidth}
\begin{flushright}
\qrcode[height=1.5cm]{https://cocalc.com/share/projects/0bcdf659-4413-4a53-be1f-ea354707091c}
\end{flushright}
\end{minipage}

\end{document}
